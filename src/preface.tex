\section*{Acknowledgements}

I'd like to express my gratitude to my supervisor Prof. Wolfgang Ketter who has invited me to join the research
community around PowerTAC and in doing so showed me an exciting new field of research. This gratitude needs to be
extended to John Collins and Nastaran Naseri who have answered many questions of mine and guided me during this work.

Thanks to any contributor of the numerous Open Source projects for sharing their work free of charge that allowed me to
bring together PowerTAC and current RL research.

Thanks to my parents Gudrun and Heiko for sparking and fostering my curiosity and for allowing me to pursue my education for years while
always supporting me and believing in me. 

Thanks to my partner Giorgia for supporting me mentally throughout these months and helping me survive Naples, my home
for the duration of this work and a city with many interesting and confusing customs. I must also thank her for
introducing me to so many fantastic Pizzaoli of Naples for their creative and nourishing Pizze that fed me
week after week. 

And now that I am back in Germany, I should probably also thank the creative mechanics that repaired my overloaded car
after breaking down somewhere in the foothills of the alps, 4 days before the deadline on my way back to Cologne. It
takes a special kind of character to fix a 22 year old Toyota on a saturday morning without any spare parts for some
young academic trying to make his way up north to complete his graduate studies. 

%This thesis was planned and discussed in the winter of 17/18. On February 1st, the work phase of six months started.
%Within these six months, I discovered many previously unknown or unforeseen complexities. These include the
%communication technologies developed to permit a complete python based broker and a large variety of API approaches
%within the RL agent libraries currently available. I invested a significant amount of effort into the
%development of the required components,and I always intended to build something that may be reused in the future instead of
%being discarded after my thesis was graded. This lead me to the decision of implementing a best practice based
%communication instead of a minimal approach and lead me to try to write my python code in a way that will let
%future broker developers reuse it as a framework for their broker implementations. 
%
%Why not just write another broker in Java? I believe PowerTAC answers an important question of our time. But I also
%believe there are not enough people working on this field and it doesn't receive the attention it should. Thousands of
%researchers and those who want to become one are working on getting AI agents to become better at Atari games or playing
%Doom. While the underlying technology advancements are fantastic, the application area is of no use to humanity. I
%wanted to apply these new technologies to a problem that matters and do so in a way that will create artifacts that
%others can build upon to outperform my solutions quickly. I wanted to create a bridge between the researchers of RL
%implementations of recent years and their large community and the exciting field of energy markets. PowerTAC offers
%another "game" to play with, another environment to let agents compete in. But it is an environment which actually
%generates value when explored and improved.
%
%As of July, I was not able to complete my research question and reach the intended target of evaluating a variety of
%neural network architectures that let a RL learn from other agents in its environment. Because of university
%regulations, changing a thesis title is not permitted. And while my research question was not answered, I believe I
%still contributed something valuable to the PowerTAC community. With my implementation, current state-of-the-art neural
%network algorithms and especially reinforcement agent implementations can be used to act in the PowerTAC competition.
%Any interested researcher with python skills can easily join the competition. And while I was not able to create a well
%performing broker in time and compete with the current participants of the competition, it is nonetheless now possible
%for others to work on a broker that deploys neural network technologies and to focus on the core problems of RL learning
%problems: Environment observation filtering, NN input preprocessing, reward function definition, neural network
%architecture experimentation etc. Using the created Docker images, developers are quickly able to start a competition
%with multiple brokers and future participants may be encouraged to adopt the Docker based distribution of their agents
%to include more advanced technologies in their broker implementations without placing a burden on others to manage these
%dependencies.  The new communication layer may be adopted by the competition maintainers to improve performance and to
%enable other platforms to be used for writing brokers.   
%
%When reading the thesis, please be aware that the title does not match the contents as one would expect. Adding a simple
%"Towards" at the beginning of the title would make it a perfect fit again. Unfortunately, I fell into the same trap that
%many software engineers and entire project teams fall into: Underestimating the complexity of the project which leads to
%either loss in quality, time overruns or budget overruns. I chose quality of the work I completed over making it work
%once but being useless for anyone else afterwards. I hope the thesis is still valuable to anyone who reads it and maybe
%upcoming graduate theses will continue where I left off. 
