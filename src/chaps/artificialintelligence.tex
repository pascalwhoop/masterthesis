\section{Artificial Intelligence}%
\label{sec:artificial_intelligence}

The field of \ac {AI} is, in comparison to Information Technology, both old and yet quiet contemporary. In the middle of
the 20th century, Alan Touring introduced the \emph{Turing Test} which, in essence, tests the ability of a human to tell
if answers to its questions are given by a machine or a human. With the advent of computers around the same time,
research has started to aim for artificial intelligence. Generally though, clearly defining \ac {AI} in a single sentence is hard.
\citet{russell2016artificial} structures historical definitions along two dimensions: The grade of how \emph{human} a system is \emph{thinks} or
\emph{behaves} and how \emph{rational} it thinks or behaves. These four directions are all pursued by researchers. In
this thesis, the goal of \emph{acting rationally} is most appropriate. 
sub fields of research in the larger field of \ac {AI}. 

%TODO prettify
\begin{table}[] 
    \renewcommand{\arraystretch}{2.5}
    \centering
    \begin{tabular}{p{0.45\textwidth}|p{0.45\textwidth}} 
        \textbf{Thinking Humanly}: The goal of creating machines with \emph{minds}
&   
        \textbf{Thinking Rationally}: Computation that can perceive, reason and act [rationally]
\\
            \textbf{Acting Humanly}: "Machines that perform functions that require intelligence when performed by people"
&
        \textbf{Acting Rationally}:  design of intelligent agents
    \end{tabular}
    \caption{Various definitions of \ac {AI} \citep{russell2016artificial}  }
    \label{tab:ai_definitions}
\end{table}

Today, some 70 years later, \ac {AI} is again extensively discussed by both researchers and public media
\citep[p.24ff.]{russell2016artificial, arulkumaran2017brief}. The reasons for this are diverse but it can be argued that
the combination of readily available computing power through cloud computing and advances in the mathematical
underpinnings have allowed for fast-paced advances in recent years. Also, the currently very popular \acf {NN}
architectures often require large amouts of data to learn which have lately been readily available for companies and
:esearchers through the adoption of online technologies by the majority of the population
\citep[p.27]{russell2016artificial}.
