The previous chapters have introduced concepts of \ac {SL} , \ac {NN}, backpropagation and \ac {RNN} for time-embedded
learning tasks. \ac {RL} can be described as an intersection between supervised and unsupervised learning concepts and
Deep \ac {RL} is the usage of \ac {NN}, especially those with many layers, to perform \ac {RL}.

On the one hand \ac {RL}  does not require large amounts of labelled data to generate successful systems which is
beneficial for areas where such data is either expensive to aquire or difficult to clearly label. On the other hand it
requires some form of feedback. Generally, \ac {RL} \emph{agents} use feedback received from an \emph{environment}.  The
general principle of \ac {RL} therefore includes an agent and the environment in which it performs actions. The function
that determines the action $a$  taken by the agent in a given state $s$ is called its policy, usually represented by
$\pi$.  The environment reacts to the actions of the agent by returning new states $s'$ which are evaluated and a
corresponding reward $r$ is given to the agent. The reward gives the agent information about how well it performed 
\citep[p.830f.]{russell2016artificial}.

%TODO still up to date with the following subsections?
This chapter will first introduce the concepts of a \ac {MDP}, then introduce different concepts of \ac {RL} agents,
describe approaches to encourage exploration of its options and finally describe how \ac {NN} can be used to create
state-of-the-art agents that can solve complex tasks. The majority of the chapter is based on
chapters 17 and 21 of \citet[]{russell2016artificial} unless otherwise marked. 

\subsection{Markovian Decision Processes}%
\label{ssub:markovian_decision_processes}

A common model describing the conceptual process of states and actions followed by new states and new actions of an
agent and its environment is called a \acf {MDP}. In fact, \ac {RL} is an approach for solving such \ac {MDP} problems
optimally
\footnote{Although \ac {RL} can also be applied to non-sequential decision problems, the field has largely focused on
sequential problems}.

A \ac {MDP} is usually defined by the following components:

\begin{itemize}
	\item $\mathcal{A}$: Finite set of allowed actions  
	\item $\mathcal{S}$: Finite set of states
	\item $P(s' \mid s,a) \forall s \in \mathcal{S}, a \in \mathcal{A}$: Probability of transitioning from state
	$s$ to state $s'$ when action $a$ is taken
	\item $\gamma$: Discount factor for each time step, discounting future rewards to allow for long-term and
		short-term focus
	\item $R(s)$: Reward function that defines the reward received for transitioning into state $s$
\end{itemize}

To solve such a problem, an agent needs to be equipped with a policy $\pi$ that allows for corresponding actions to each
of the states. The type of policy can further be distinguished between \emph{stationary} and \emph{nonstationary}
policies. The former type refers to policies that recommend the same action for the same state independent of the
time stetime step. The latter describes those policies which are trying to solve non-finite state spaces and where an
agent might therefore act differently once time becomes scarce. However, also infinite-horizon \ac {MDP} can have
terminal states which conceptually mean that the process has ended.

A more complex form of \ac {MDP} is the \ac {POMDP} which involves agents basing their actions on a belief of the
current state. As the later practical application to the \ac {PowerTAC} competition however can be mapped to a \ac {MDP}
where the transition probability implicitly represents the partial observability \citep{tactexurieli2016mdp}, this will not be discussed. 

\subsection{Bellman Equation}%
\label{ssub:bellman_equation}

The Bellman Equation offers a way to describe the utility of each state in an \ac {MDP}. For this, it defines the
utility of a state as the reward for the current state plus the sum of all future rewards discounted by $\gamma$. 

\[
U(s) = R(s) + \gamma \max_{a\in\mathcal{A}(s)} \sum_{s'}{P(s' \mid s,a)U(s')}
%TODO numbers on equation?
\]

In the above equation, the \emph{max} operation selects the optimal action in regard to all possible actions. The
Bellman equation is explicitly targeting \emph{discrete} state spaces. If the state transition graph is a cyclic graph
the solution to the Bellman equation requires some equation system solving. That is because $U(s')$ may depend on $U(s)$
and the other way around. Further, the \emph{max} operator creates nonlinearity which, for large state spaces, becomes 
intractable quickly which is the reason for an iterative approach called \emph{Value Iteration}.

%In a discrete action space this would be a selection over all possible actions, in a continuous action space it however
%can become more complex. \ac {NN} based \ac {RL} agents simply invoke their policy network to retrieve the action which
%the agent believes it the one with the highest utility \citep{mnih2013playing}.


%As an example, an agent active in the environment of playing Super Mario may receive rewards corresponding to the game
%score. It may perform all valid moves permitted by the game and the goal is to improve its score. 

%To train an agent, the task is usually performed several times and the environment is reset after each iteration to
%allow for a new learning step.  
%TODO add atari games reference
%When thinking about such an agent, it becomes obvious that without some explicit incentive to explore new alternatives,
%it may be contempt with whatever success it achieves and then always perform the same action. To avoid this, the agent
%can either be forced to try new alternative actions (through forcing random actions in a certain percentage of cases) or
%through explicit rewards for random actions.
%\citep() %A3C TODO


%There are several forms of learning to solve a \ac {MDP} using \ac {RL} but for the purpose of this thesis I will focus
%on explaining how policy search algorithms work. Their goal is to find a good policy $\pi$ given a certain state $s$.
%Alternatives to this approach are concepts that try to learn expected future reward for future possible states, for
%available actions and others. %TODO not clean paragraph...

\subsection{Value and Policy Iteration}%
\label{sub:policy_and_value_iteration}

Value Iteration uses the Bellman equation to iteratively converge towards a correct estimation of each states utility,
assuming both the transition function $P(s' \ mid s,a) \forall s \in \mathcal{S}$ and the reward function $R(s)$ are
known to the agent.  
In the algorithm, the utility of each state is updated based on the \emph{Bellman update} rule:
\[
U_{i+1}(s) \gets R(s) + \gamma \max_{a \in \mathcal{A}(s)} \sum_{s'}{P(s' \mid s,a) U_i(s')}
\]
This needs to be performed for \emph{each} state during \emph{each} iteration. It is clear how quickly this becomes
intractable as well when $\gamma$ is reasonably close to 1, meaning that also long-term rewards are taken into
consideration. 

Practically, the agent however doesn't care much about the values of various states. It cares about making the right
decisions, using the value of states as a basis for doing so. It is often observed that the policy $\pi$ converges far
sooner than the utility estimates $U(s)$. This is the basis for the \emph{Policy Iteration} approach which alternates
between: 
\begin{enumerate}
	\item evaluating the current policy $\pi_i$ by calculating $U_i=U^{\pi_i}$, the value of each state if $\pi$ is
	executed and 
	\item improving the policy using one-step look-ahead based on $U_i$
\end{enumerate}

This process stops when the policy is no longer showing any significant improvements in respect to its loss value. It is
generally also not necessary to always apply the above operations to \emph{every} state. Instead, state values and
policies can be updated only in respect to newly discovered knowledge regarding specific states or specific actions.
This is called \emph{asynchronous policy iteration}. 

Both variants require the transition function and the reward function to be known to the agent. \ac {RL} research has
developed several methods that adapt the concepts of the two iteration algorithms for environments with the two unknown
functions. They are explained in the next sections. 
%It allows for many interesting concepts to be used such as distributed learning, using multiple agents to explore an 
%environment and learn from all their observations centrally, and  

\subsection{Temporal Difference Learning}%
\label{sub:temporal_difference_learning}

When the underlying transition function is not known, but the agent has the ability to perform many trial runs in the
environment, an empirical approach can be adapted. For this, the agent performs a number of trials where it acts
according to a (fixed) policy and observes the rewards it receives. Each string of alternating actions and observations
is called a trial
\footnote{in newer \ac {RL} literature this is also called a \emph{trajectory} \citep{proximalpolicyopt, heess2017emergence} }. 
The update rule for the utility of each state is as follows:
\[
U^\pi(s) \gets U^\pi(s) + \alpha(R(s) + \gamma U^\pi(s') - U^\pi(s))
\]

Where $\alpha$ is the learning rate and $U^\pi$ the utility under the execution of $\pi(s)$ in state $s$. This only
updates the utilities based on the observed transitions so if the unknown transition function sometimes leads to
extremely negative rewards through rare transitions, this is unlikely to be captured. However, with sufficiently many
trials, these rare probabilities will be adequately represented in the utilities for the states. If $\alpha$ is
continuously reduced appropriately, this will converge to the correct value. 

\subsection{Exploration}%
\label{sub:exploration}

The above learning approach has one weakness: It is only based on observed utilities. If $\pi$ follows the pattern of
always choosing the action that leads to the highest expected $U_{i+1}$, i.e.
\[
\pi(s) = \max_{a \in \mathcal{A}(s)}P(s' \mid s, a)U(s')
\]
then it will never explore possible alternatives and will very quickly get stuck on a rigid action
pattern mapping each state to a resulting action. To avoid this, the concept of \emph{exploration} has been introduced.
There are many approaches to encourage exploration. The simplest is to define a factor $\epsilon$ which defines the
probability of choosing a random action at each step. 
A more advanced variant is to add a term to the loss function that
corresponds to negative entropy of the policy $-\beta H(\pi(a \mid s ))$ where $H$ measures the entropy of a series of
actions. This encourages randomness in the policy but it permits the policy function to determine how this randomness 
gets to occur \citep{schmitt2018kickstarting}. This entropy based loss also automatically regulates itself: When the
agent is not at all capable of choosing rewarding actions it reduces its loss through high entropy choices, i.e. lots of
exploration. Once the agent finds actions for certain states that lead to high rewards, choosing other random actions
negatively outweighs following the best action. Therefore, it becomes less random and the entropy reduces. If $\beta$ is
progressively lowered, the impact on the loss is also progressively lowered, allowing the agent to continuously improve
its loss despite less exploration. 
Another alternative is the positive weighting of actions in states that have not been tried yet, essentially giving such
actions an optimistic prior as if they promise higher rewards than the already explored regions. This is easy to
implement for small, discrete state and action spaces but more complex for continuous spaces. 

% Q Learning and Deep-Q learning
% Policy Gradient


%\subsection{Policy Search} 
%\subsection{Deep Reinforcement Learning}
%%TODO paper deep \ac{RL} > mixing R.L and deep \ac{NN} 
%\subsection{Proximal Policy Optimization OR(TBD) Deep Q learning}
\subsection{Q Learning}%
\label{sub:q_learning}

In Section~\ref{sub:policy_and_value_iteration} I have already described how to learn the values of states, given an
action. This action can also be derived from a policy function.  In the case of an agent that wants to learn its policy
(i.e. learn what a good policy is), this becomes problematic if the transition function is not known. An alternative
model is called \emph{Q-Learning} which is a form of Temporal Difference Learning. It learns an action-utility value
instead of simply the values. The relationship between this \emph{Q-Value} and the former value of a state is simply 
\[
U(s) = \max_{a}Q(s,a)
\]
so the value of a state is that of the highest Q-Value. The benefit of this approach is that it does not require a model
of how the world works, it therefore is called a \emph{model-free} method. The update rule for the Q-Values is simply
the Bellman equation with $U(s)$ and $U(s')$ replaced with $Q(s,a)$ and $Q(s',a')$ respectively. 

The update rules for the Q-Value approach are related to the Temporal Difference Learning rules but include a $\max$
operator
\[
Q(s,a) \gets Q(s,a) + \alpha(R(s) + \gamma \max_{a'}Q(s', a') - Q(s,a))
\]
An alternative version is the reduction of the above equation by removing the $\max$ operator. This results in the
\emph{actual} action being considered instead of the one that the policy believes to be the best. Q-Learning is
\emph{off-policy} while the latter version, called \ac {SARSA}, is \emph{on-policy}. The distinction has a significant
consequence: While Q-Learning may be used to learn the Q-Values from recorded state-action pairs, \ac {SARSA} requires
the action taken to be derived from the current policy function. 

\subsection{Policy Search and Policy Gradient Methods}%
\label{sub:policy_search_and_policy_gradient_methods}

These two approaches are possibly the simplest of the \ac {RL} algorithms. In its simplest form, policy search requires
the algorithm to start with an initial policy and then adapt this policy until no further gains can be made. While the
concepts is simple, it may lead to significant performances, if the \emph{choices regarding what to change} are made
wisely. If the policy is just randomly changed, the results will be equally random. If the policy is changed depending
on a good interpretation of the environments responses however, this method can offer good performance without the need
to have a model of how the world works. Such an agent simply takes the current state $s$ as input and uses its policy to
determine an output action $a = \pi(s)$. The value of a policy is noted as $\rho(\theta)$.

For simplicity, I will assume actions derived from a policy to be continuous as both the later application relies on
such actions and because the analysis of policy search algorithms becomes more complex in discrete action spaces. When
both the policy and the environment are deterministic and without noise, policy search algorithms are actually extremely
effective. The agent can repeat actions in the equivalent states several times, adapting its policy parameters $\theta$ by
small values and determine the empirical gradient values which allow the agent to perform hill-climbing in the policy
function. This will converge to a local optimum, hence simply trying different actions allows the agent to improve its
performance as long as the local optimum has not been reached. 

In real world scenarios however, environments (and also policies) are commonly stochastic. Changing the policy
parameters $\theta$ by a very small value and comparing results of two instances of executing the policy may lead to
strong variations in the reward due to the stochasticity of the environment and therefore the noise in the reward
signal. This is a common problem of statistics and the typical answer is to increase the number of trials until
statistical significance can be reached. But this is often impractical for real world problems and also not the best
approach. 

The general idea of modern policy gradient methods therefore follows an approach of using a different function as an
estimator for the gradient of the policy in a given configuration. A common approach is to use an advantage function
$\hat{A}_t$ to create an estimator for the policy gradient:


\begin{equation*}
\hat{g} \ =\ \hat{\mathbb{E}}_{t} \ \left[ \nabla _{\theta }\log \pi _{\theta }( a_{t} \ \mid s_{t})\hat{A}_{t}  \right]
\end{equation*}

where $\hat{A}_t$ describes the advantage of taking one action over another in a given state. It can therefore be
described as an \emph{actor-critic architecture}, because $A(a_t, s_t) = Q(a_t,s_t) - V(s_t)$, meaning that the
advantage value is equivalent to the difference in the estimated value of the state itself and the value of performing
a specific action (derived from the policy) in that state \citep{mnih2016asynchronous} 

\subsection{Deep Learning in Reinforcement Settings}%
\label{sub:deep_learning_in_reinforcement_settings}

The previous sections have outlined the conceptual approaches for designing learning agents based on various approaches
for what is essentially a system that tries to act intelligently in respect to its environment. Especially in recent
years, many breakthroughs have been driven by using \ac {NN} in \ac {RL} settings. \ac {NN} have proven effective as
parameterized Q-Value estimators, state-value estimators and as policy functions. Most approaches suffer similar
problems: Data efficiency in respect to the trial number required to learn a desired skill, scalability and robustness
\citep{proximalpolicyopt}. The reasons for these challenges are obvious: The agents receives minimal feedback and it has
a hard time mapping its received reward to specific alterations in its behavior (aka credit assignment problem)
\citep{arulkumaran2017brief}.  

The research has shown many approaches to alleviate these shortcomings: Inverse Reinforcement Learning allows the
learning of a policy by imitating a trusted expert, allowing faster learning rates through clear signals
\citep{NG2000InvReinf}. \citet{brockman2016openai} have created the \emph{gym} which allows for coherent benchmarking of
various approaches against a common set of challenges. \citet{hafner2017agents} have created a setup that allows for
massive parallel processing of several environments which all contribute to the improvement of a central policy
function. The implemented algorithm uses \ac {NN} and an advanced form of the advantage based policy methods introduced
earlier. By this time, the research groups were usually referring to their agents learning progress in the range of
millions of time steps \citep{proximalpolicyopt}. One common argument for the benefit of \ac {AI} is the ability to
transfer knowledge learned by one agent to many agents. The structure of \ac {NN} however makes it difficult to transfer
knowledge between agents with varying hyperparameters. The weights for the neurons cannot simply be copied between
networks with different structures and due to the complexity of the systems, learning them again from scratch is
resource intensive.

\citet{matiisen2017teacher} have introduced a concept that helps agents learn faster by breaking complex challenges down
into several simpler sub tasks, similar to how humans are taught in educational institutes. This allows researchers to
quickly get new generations of agents up to speed with their predecessors. Another approach to solve this problem of
repetitive learning has been introduced by \citet{schmitt2018kickstarting}. In their setup, the newly created agent
transitions from trying to act similar as its \emph{teacher agent} towards trying to improve its performance
independently. To achieve this, the student agent includes a term that describes the difference between its action and
the action its teacher would have taken.

In summary, many tweaks to the core concepts allow for improvements in the challenges outlined before. Faster learning given limited
ressources through bootstrapping, improving wall time by leveraging large-scale architectures through and
parallelization, transfering knowledge from (human) experts through inverse \ac {RL} etc. A rich landscape of tools is
in rapid development and to construct an effective agent, it is important to leverage both the specific problem domain
structure and the available resources. 
