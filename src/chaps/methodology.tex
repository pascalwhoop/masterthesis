First I'll perform a literature research into the fields of \ac{AI}, \ac{RL} and Animal Cognition. In the field of AI
it's sub fields of \ac {SL}  and \ac {UL}  will be introduced. Here I will focus on the area of \ac{NN} and a form of \emph{learning}  called Backpropagation. In the field of \ac{RL} I will focus on the \ac{MDP} framework as well as the \ac{POMDP} subclass. 
Next follows an introduction of the recent research in using \ac{NN} in \ac{RL} settings to allow for what is now called Deep Reinforcement Learning. This field has seen tremendous success in recent research, allowing for agents that successfully play 
%TODO ref
Atari games and the game Go on superhuman levels of performance.
Lastly, I will introduce the specific learning model that will also be used in the development of my own agent, called %TODO
.
%some explanation about the field of Animal Cognition and animal learning

After having introduced the basic research of \ac{AI} and \ac{RL}, I will summarize the state of research of Animal Cognition, which focuses on how animals and humans learn, act and remember in their environment. Since humans and animals are the only known form of intelligent life to us as of today, it is intuitive why exploring the exact workings of these examples might help in better understanding how to artificially create intelligence. It is also a basis of the thesis, as many animals show forms of social learning, concepts of teaching and learning through observation.
%TODO ref 

Following the theoretical foundations of the work I will introduce the concept of competitive simulations in research and explain the \ac{PowerTAC} competition, it's parts and how agents (called brokers in the context of \ac{PowerTAC}) make decisions. 

Next, I will explain how I implemented an agent in the context of \ac{PowerTAC} that is able to learn from other agents
in the environment first, before engaging in solo learning through applying techniques of \ac{SL}  combined with heuristics for recognizing other agents skill level %TODO what heuristics? need to develop
. For this, I will introduce the tools used as well as the overall architecture of the agent software, followed by the mathematical concepts that are applied to the problem. Afterwards some results of the agent's performance in the competition are described and analyzed.

Finally, a conclusion is drawn and the limits and weaknesses as well as recommended further research is discussed.

