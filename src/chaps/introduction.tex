TODO Intro comes at the end

% intro structuring basing on style from https://explorationsofstyle.com/2013/01/22/introductions/
%Intro short:
% - global warming, lots of problems
% - reinvent the energy grid, lots of changes to the structure
%   - very difficult to construct such a highly complex, globally spanning, must-never-fail system
% - recent developments of of A.I. and machine learning
% - combine the two

%Intro long
% - energy grids of the future background research (PTac)
%     - key components of such an intelligent agent (prediction, actions --> \ac {SL} and \ac {RL} )
% - research in \ac {SL} and \ac {RL} has seen huge improvements in recent years, thanks to \ac {NN} 
% - agents/brokers in the field of PTac haven't been seeing much of these improvements
% - also an issue of "adopting what has been learned by previous agents (transfer learning issues)"
% - 
% - 
%TODO reiterate over the research question from my proposal. Is it still applicable?

%-------------------------------------------------------------------------------  
%\emph{Can \ac{RL} agents learn from other agents in the environment? If so, how? Can mutual learning and imitation
%allow for boosted performance of reinforcement algorithms within a competitive simulation environment?}
%------------------------------------------------------------------------------- 

Global warming is a key challenge of the near and medium future. Without proper action, entire continents will see
%TODO END

Global warming, if not combated, will change the face of the planet. Billions will be impacted, entire coastlines will
be changed and cities all over the global will have to either be retrofitted to handle sub-sea level positioning or
abandoned and relocated. (global warming report)


One key component to avoid such disastrous effects is the reinvention of the energy systems of the world. While
appliances on an individual level need to become ever more efficient, globally it is necessary to shift the
transportation sector towards renewable energy sources.
Solar and wind
are required. But The future of energy is difficult (--> MISQ paper argumentation line)

Smart grids need decentralized intelligence where appliance level evaluation of the grid status impacts how energy is
consumed. When such intelligence shifting is happening towards the \emph{edge} of the grid, it can be intelligent to
introduce intermediate broker entities that mediate between the two extremes, the end-consumers and the wholesale
market.

At the same time, current developments in AI and machine learning allow for highly sophisticated learning machines that
can help manage complex tasks and systems. (citing some sexy AI papers)

Bringing these two developments together, it is intuitive to apply some of the recently developed technologies of 
\ac {AI} research to solve the coordination issues of contemporary, frankly crude energy networks. 
